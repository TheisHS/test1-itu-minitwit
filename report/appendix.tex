\section{Appendix}
\renewcommand{\thesubsection}{\Alph{subsection}}

\subsection{Choice of programming language}
\label{app:programming_language_choice}
\begin{table}[H]
    \centering
    \footnotesize 
    \begin{tabularx}{\textwidth}{|L|L|}
    \hline
    \textbf{Feature Category} & \textbf{Description} \cr \hline
    Database Operations & Connect to the database \cr \cline{2-2}
                         & Create the database tables \cr \cline{2-2}
                         & Query the database \cr \cline{2-2}
                         & Get the user id \cr \cline{2-2}
                         & Format a timestamp for display \cr \cline{2-2}
                         & Get a gravatar image for an email \cr \hline
    Request Handling    & Before a request, it connects to the DB and gets the user, and after a request it closes the connection to the DB. \cr \hline
    UI                  & A page showing the public timeline, including all public messages. \cr \cline{2-2}
                         & A page showing the timeline for the logged-in user. If no user is logged in, redirect to the public timeline. The private timeline includes all public messages, and the messages from the people the logged-in user follows. \cr \cline{2-2}
                         & A page for user profiles. If the user is followed, display a different message ("You follow this user"/"You don't follow ...") \cr \hline
    Routes              & A route to follow other users \cr \cline{2-2}
                         & A route to unfollow other users \cr \cline{2-2}
                         & A route for adding messages \cr \cline{2-2}
                         & A route to register \cr \cline{2-2}
                         & A route to login \cr \cline{2-2}
                         & A route to logout \cr \hline
    \end{tabularx}
    \caption{Feature mapping of ITU Minitwit}
    \label{tab:feature_mapping}
\end{table}

\begin{table}[H]
    \centering
    \footnotesize 
     \begin{tabularx}{\textwidth}{|L|L|L|L|L|}
    \hline
        ~ & Python3 Flask & Crystal Kemal & Ruby Sinatra & Golang \cr\hline
        Our experience & Moderate experience & No experience & No experience & Moderate experience \cr\hline
        Types & Dynamically typed & Statically typed & Dynamically typed & Statically typed \cr\hline
        Performance & Moderate performance & High performance & Moderate performance & High performance \cr\hline
        SQLite support & yes & yes & yes & yes \cr\hline
        Middleware support* & yes & yes & yes & yes \cr\hline
        Release date & 2010 & 2016 & 2007 & 2011 \cr\hline
        Deployment & Deployed using virtual environments to manage dependencies. & Compiled into a single binary executable with all its dependencies. & Requires the presence of the Ruby runtime environment and dependencies. & Can be deployed as single binary executables. \cr\hline
    \end{tabularx}
    \caption{Programming language strengths}
    \label{tab:programming_language_choice}
\end{table}

* Middleware in web APIs is used as a design pattern to intercept and manipulate HTTP requests.
\url{https://azure.microsoft.com/en-us/resources/cloud-computing-dictionary/what-is-middleware}

\newpage
\subsection{List of Technologies and Tools}
\label{app:technologies_and_tools}

\begin{multicols}{2}
    \begin{itemize}
        \item \textbf{Go}: A statically typed, compiled programming often used for backend development.
        \item \textbf{Go/Gorilla}: A toolkit for Golang, that provides packages for building web applications. Used for routing and session management.
        \item \textbf{SQLite}: A lightweight SQL database engine that is ideal for embedded database applications. Used in initial setup and for testing.
        \item \textbf{Docker}: A platform for shipping and running applications, ensuring consistency across different environments. Containerization is used deliberately throughout our system and in delivery.
        \item \textbf{Docker Swarm}: A container orchestration tool, included in Docker by default, to manage a cluster of Docker nodes as a single virtual system.
        \item \textbf{DigitalOcean}: A cloud infrastructure provider offering scalable compute and storage solutions. Used for deploying and managing the application and database.
        \item \textbf{Prometheus}: An open-source monitoring toolkit used for monitoring metrics in our cloud environments.
        \item \textbf{Grafana}: An open-source analytics and monitoring platform that integrates various data sources to visualize our data and metrics.
        \item \textbf{PostgreSQL}: A open-source object-relational database system, known for extensibility and speed. Used after the transition from SQLite managed by DigitalOcean.
        \item \textbf{Promtail}: An agent that ships log file to Loki and part of the Grafana logging stack. Used for gathering logs.
        \item \textbf{Loki}: A log aggregation system designed to be scalable. Used with Grafana for log querying and visualization.
        \item \textbf{Nginx}: A high-performance web server and reverse proxy server. Used for load balancing.
        \item \textbf{CertBot}: A open-source tool for automatically using Let's Encrypt certificates to enable HTTPS on web servers.
        %\item \textbf{Terraform}: An infrastrcucture as code tool that allows users to define and provision data center infrastructure using a high-level configuration language.
        \item \textbf{CircleCI}: A continuous integration and delivery platform that automates the build, test and deployment processes of our project. 
        \item \textbf{CodeClimate}: A platform that provides code review, offering insights into code quality, maintainability and test coverage, helping us ensure high standards and to improve code health.
        \item \textbf{Github}: Version Control Management
        \item \textbf{Hadolint}: A lint-tool that analyzes Dockerfiles to find common issues.
        \item \textbf{Shellcheck}: A shell script analysis tool that identifies syntax errors and common issues.
        \item \textbf{Sonarcloud}: A cloud-based code quality and security service that analyzes code to detect bugs and vulnerabilities.
        \item \textbf{Vagrant}: A tool for building and managing virtualized development environments. Used initially for VM instantiating.
    \end{itemize}
\end{multicols}

\newpage
\subsection{Choice of CI tool}
\label{app:ci_tool_choice}
As our code is stored on Github, we've eliminated Gitlab CI as as an option. Travis CI is only free for a single month (for students), and we've also eliminated this.

\begin{table}[H]
    \centering
    \begin{tabularx}{\textwidth}{|L|L|L|}
    \hline
        Github Actions & CircleCI & Our considerations \cr\hline
        Is free ... "Cheapest for people with public repositories" & 3000 minutes for free pr month & We have a public repo. \cr\hline
        Runs full pipeline automatically & Can be paused and wait for human interaction & We don't have a usecase for needing human intervention before deploying if the code passes all the tests we stup, and the CircleCI feature (even though nice) is not needed. \cr\hline
        More than CI/CD - can also automate manual tasks like generating changelogs or versioning releases & Only CI/CD, but specialised in this. & We only need CI/CD for now. \cr\hline
        Slower than CircleCI & Faster than Github Actions & Do we need speed? \cr\hline
        Only Windows, MacOS and Linux & Every operating system & We only need Linux \cr\hline
        Configuration can be split in mulitple files & Single file configuration & Cleaner setup with GHA? \cr\hline
        Docker support is still a bit buggy on GHA, and works only with Linux. & CircleCI has perfected its Docker support over the years to make it (almost) the de-facto environment for running builds. & We will use Docker, but still only Linux. \cr\hline
        More granular control by exposing all commands. Complexity increase. & Less complex, has built in commands for often-used services. Less control. & We don't know what we need yet - so maybe more control is nice, but it being easy is also nice. \cr\hline
    \end{tabularx}
    \caption{CI tool strengths.}
\end{table}

\newpage
\subsection{Static code analysis}
\label{app:static_analysis}
Here we provide a snapshot of our static code analysis tools from May 19, 2024.
\begin{figure}[H]
    \centering
    \includegraphics[width=\linewidth]{images/codeclimate_snapshot_19052020241251.png}
    \caption{Code Climate summary 19/05/2024.}
    \label{fig:codeclimate_snapshot}
\end{figure}
\begin{figure}[H]
    \centering
    \includegraphics[width=\linewidth]{images/codeclimate_trend_snapshot_19052020241251.png}
    \caption{Code Climate Technical Debt Trends 19/05/2024.}
    \label{fig:codeclimatetrends_snapshot}
\end{figure}
\begin{figure}[H]
    \centering
    \includegraphics[width=\linewidth]{images/sonarcloud_snapshot_19052020241251.png}
    \caption{SonarCloud summary 19/05/2024.}
    \label{fig:sonarcloud_snapshot}
\end{figure}

