\section{Appendix}
\renewcommand{\thesubsection}{\Alph{subsection}}

\subsection{Choice of programming language}
\label{app:programming_language_choice}

\begin{table}[H]
    \centering
    \footnotesize 
     \begin{tabularx}{\textwidth}{|L|L|L|L|L|}
    \hline
        ~ & Python3 Flask & Crystal Kemal & Ruby Sinatra & Golang \cr\hline
        Our experience & Moderate experience & No experience & No experience & Moderate experience \cr\hline
        Types & Dynamically typed & Statically typed & Dynamically typed & Statically typed \cr\hline
        Performance & Moderate performance & High performance & Moderate performance & High performance \cr\hline
        SQLite support & yes & yes & yes & yes \cr\hline
        Middleware support* & yes & yes & yes & yes \cr\hline
        Release date & 2010 & 2016 & 2007 & 2011 \cr\hline
        Deployment & Deployed using virtual environments to manage dependencies. & Compiled into a single binary executable with all its dependencies. & Requires the presence of the Ruby runtime environment and dependencies. & Can be deployed as single binary executables. \cr\hline
    \end{tabularx}
    \caption{Programming language strengths}
    \label{tab:programming_language_choice}
\end{table}

* Middleware in web APIs is used as a design pattern to intercept and manipulate HTTP requests.
\url{https://azure.microsoft.com/en-us/resources/cloud-computing-dictionary/what-is-middleware}


\newpage
\subsection{Choice of CI tool}
\label{app:ci_tool_choice}
As our code is stored on Github, we've eliminated Gitlab CI as as an option. Travis CI is only free for a single month (for students), and we've also eliminated this.

\begin{table}[H]
    \centering
    \begin{tabularx}{\textwidth}{|L|L|L|}
    \hline
        Github Actions & CircleCI & Our considerations \cr\hline
        Is free ... "Cheapest for people with public repositories" & 3000 minutes for free pr month & We have a public repo. \cr\hline
        Runs full pipeline automatically & Can be paused and wait for human interaction & We don't have a usecase for needing human intervention before deploying if the code passes all the tests we stup, and the CircleCI feature (even though nice) is not needed. \cr\hline
        More than CI/CD - can also automate manual tasks like generating changelogs or versioning releases & Only CI/CD, but specialised in this. & We only need CI/CD for now. \cr\hline
        Slower than CircleCI & Faster than Github Actions & Do we need speed? \cr\hline
        Only Windows, MacOS and Linux & Every operating system & We only need Linux \cr\hline
        Configuration can be split in mulitple files & Single file configuration & Cleaner setup with GHA? \cr\hline
        Docker support is still a bit buggy on GHA, works only with Linux. & CircleCI has perfected its Docker support over the years to make it (almost) the de-facto environment for running builds. & We will use Docker, but still only Linux. \cr\hline
        More granular control by exposing all commands. Complexity increase. & Less complex, has built in commands for often-used services. Less control. & We don't know what we need yet - so maybe more control is nice, but it being easy is also nice. \cr\hline
    \end{tabularx}
    \caption{CI tool strengths.}
\end{table}

\newpage
\subsection{Static code analysis}
\label{app:static_analysis}
Here we provide a snapshot of our static code analysis tools from May 19, 2024.
\begin{figure}[H]
    \centering
    \includegraphics[width=\linewidth]{images/codeclimate_snapshot_19052020241251.png}
    \caption{Code Climate summary 19/05/2024.}
    \label{fig:codeclimate_snapshot}
\end{figure}
\begin{figure}[H]
    \centering
    \includegraphics[width=\linewidth]{images/codeclimate_trend_snapshot_19052020241251.png}
    \caption{Code Climate Technical Debt Trends 19/05/2024.}
    \label{fig:codeclimatetrends_snapshot}
\end{figure}
\begin{figure}[H]
    \centering
    \includegraphics[width=\linewidth]{images/sonarcloud_snapshot_19052020241251.png}
    \caption{SonarCloud summary 19/05/2024.}
    \label{fig:sonarcloud_snapshot}
\end{figure}

