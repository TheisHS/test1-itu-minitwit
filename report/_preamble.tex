\usepackage[a4paper,margin=1in,bottom=1.2in]{geometry}
\usepackage[utf8]{inputenc}
\usepackage{
    amsmath, % \begin{align}
    amssymb, % math symbols såsom \rightarrow 
    graphicx, % indsæt billeder
    wrapfig, % indsæt figurer ved siden af tekst
    float, % vælge placering af figures med [H]
    enumitem, % ændre label på enumeration
    fancyhdr, % header og footer
    colortbl, % farve tabellers celler
    tabularx, % mere kontrol over tabeller
    listings, % kodebokse
    nameref, % referere til mere end bare label-number
    caption,
    upquote,
    multirow,
    multicol
} 
\usepackage[dvipsnames]{xcolor}
\PassOptionsToPackage{hyphens}{url}\usepackage[hidelinks]{hyperref} % embed links
\usepackage[bottom]{footmisc} % put footnotes as much to the bottom as possible.
\usepackage[backend=bibtex]{biblatex}
\addbibresource{ref.bib}% or \bibliography{bibfilename}

\renewcommand{\ttdefault}{pcr}


\title{Devops 2024}
\author{Carl (carbr@itu.dk) \\ Stender (ches@itu.dk) \\ Nadja Brix Koch (nako@itu.dk) \\ Theis Helth Stensgaard (thhs@itu.dk)}
\date{}

% Sideopsætning
\pagestyle{fancy} % Bred side med header og footer
\fancyhead[LR]{} % Headeren er nu tom
\renewcommand{\headrulewidth}{0pt} % Ingen streg under headeren
\renewcommand{\footrulewidth}{0pt} % Ingen streg under footeren
\setcounter{tocdepth}{2} % Hvor meget skal med i tocs

% Tekstformattering
\setlength{\parindent}{0pt} % Ingen indryk
\linespread{1.2} % Linjeafstand
\setlength{\parskip}{1ex plus 0.5ex minus 0.2ex} % Et lille linjeskip i stedet for indryk

\newcommand{\dashline}{- - - - - - - - - - - - - - - - - - - - - - - - - - - - - - - - - - - - - - - - - - - - - - - - - - - - - - - - - - - -}
%\newcommand{\dashline}{-- \;\;-- \;\;-- \;\;-- \;\;-- \;\;-- \;\;-- \;\;-- \;\;-- \;\;-- \;\;-- \;\;-- \;\;-- \;\;-- \;\;-- \;\;-- \;\;-- \;\;-- \;\;-- \;\;-- \;\;-- \;\;-- \;\;-- \;\;-- \;\;-- \;\;-- \;\;-- \;\;-- \;\;-- \;\;-- \;\;-- \;\;-- \;\;--}

\newcommand{\bibref}[1]{[\ref{#1}]}

% Tabeller, afhængig af tabularx package
\newcolumntype{R}{>{\raggedleft\arraybackslash}X}
\newcolumntype{N}{>{\raggedright\hsize=4cm}X}
\newcolumntype{L}{>{\raggedright}X}
\newcolumntype{C}{>{\centering\arraybackslash}X}
    
% Farver
\definecolor{mygrey}{rgb}{0.5,0.5,0.5}

% farvet tekst, afhængig af colorx package
\newcommand{\red}[1]{\textcolor{red}{#1}}

\captionsetup{justification=raggedright,singlelinecheck=false,font={color=mygrey,small}}

% Til kodebokse, afhængig af listings package
\lstset{
  frame=none,
  basicstyle=\small\ttfamily,
  language=Java,
  aboveskip=3mm,
  belowskip=-2mm,
  showstringspaces=false,
  backgroundcolor=\color{lightergray},
  numbers=left, 
  numberstyle=\tiny\color{mygrey},
  breakatwhitespace=true,
  framexleftmargin=6pt,
  framextopmargin=2pt,
  framexbottommargin=2pt, 
  frame=tb, framerule=0pt,
}
