\section{Lessons Learned Perspective}
%\textcolor{red}{Describe the biggest issues, how you solved them, and which are major lessons learned with regards to: evolution and refactoring operation, and maintenance of your ITU-MiniTwit systems. Link back to respective commit messages, issues, tickets, etc. to illustrate these. Also reflect and describe what was the "DevOps" style of your work. For example, what did you do differently to previous development projects and how did it work?}


% \subsection{Key Challenges Faced}

% \subsubsection*{Problem-Solving Approaches}
% - Quick to notify each other on Discord, discuss severity and decide course of action. 
% - If one member was more involved with an issue, they would sit together with someone not involved -> both insider knowledge of how it is set up and should work, with a fresh mind seeing it from a different perspective.



\subsection{Challenges with Evolution and Refactoring}

We experienced that when adding new technologies to our system, other parts of our system stopped working. 
%  An example of this, which had a huge impact on our project, was when trying to add Docker Swarm, we stopped receiving logs. 
When parts of our system failed, we tried getting them to work again by finding guides online that was specifically targeted at making the newly added technology and the existing 'broken` technology work together. 
This has taught us the importance of modular solutions that are open for extension and are easily refactored. 
One of the agile principles state that "Working software is the primary measure of progress``, and in line with this, we have truly experienced in this project how broken software halts progress. 


% - Importance of modular solutions open for extension and easy to refactor, we experienced that often setting up something resulted in more issues than solved issues. Some of this is also due to the way the course is structured. First adapt everything to one type of database -> refactor everything to a new database. Same exercise with making the system scalable and going from docker to docker swarm.
% - Prioritisation


\subsection{Challenges with Operations}
A challenge with our CI/CD management tool was the inability to run our automated tests until a pull request was merged to our main branch.

Running tests locally was a lengthy process as it required several docker containers being built and started. As a result, we sometimes approved pull requests with failing code. 
This resulted in situations such as commit \texttt{7661b99}\footnote{\ \url{https://github.com/TheisHS/test1-itu-minitwit/commit/7661b997afbc0de3742e7df066b37245ef7f18bd}} and \texttt{8071acb}\footnote{\ \url{https://github.com/TheisHS/test1-itu-minitwit/commit/8071acb95fa1010bb102b45b32634bb6b6790e70}}, requiring a new branch and pull request to salvage the main branch.
To solve this, we considered setting up Github Actions with the same tests as in our CircleCI workflow, but we decided it was redundant.
However, we have learned that if running tests locally is too difficult, it will not be done, highlighting the importance of automated tests.\\ 
Moreover, we found that pushing failing code to main is not as dangerous as it sounds, highlighting the importance of a fault tolerant CI/CD pipeline. Also, having the code on Github decoupled from the code on our VMs makes it safe to push code to our main branch. Ultimately this has allowed us to continuously deliver and deploy, which again is in line with the agile principles\footnote{\ "Deliver working software frequently, from a couple of weeks to a couple of months, with a preference to the shorter timescale``.}.




\subsection{Challenges with Maintenance}
\label{sec:maintenance}
% - Time consumption: bite the bullet and accept not everything can be done 100\% and if things break there might not be time to fix all of it.
% - Importance of tooling such as monitoring; Easily see if things are working properly at different layers of the system (frontend, api...), logging; Good tool for error debugging, codeclimate; 

%\subsubsection*{Monitoring and logging}
Early in this project, monitoring was limited to the official status-page\footnote{\ \url{http://206.81.24.116/status.html}}, where it was easy to see if \textit{something} was wrong, but not what it was. 
This is where logging becomes a powerful tool.
We started using a stack consisting of Promtail, Loki and Grafana, this allowed us to see that there were a lot of usernames that were not able to be found. 
However, when we started to work with Docker Swarm, our stack failed, and again it left us completely blind to the cause of the errors we experienced. 
%This taught us, that monitoring is an important tool to give you an overview of the system is up and running. 
In the end, it meant that one of our Github issues\footnote{\ \url{https://github.com/TheisHS/test1-itu-minitwit/issues/41}}, which was identified early in our process, was never solved and closed.


\subsection{Other Reflections on DevOps Practices}
%Also reflect and describe what was the "DevOps" style of your work. For example, what did you do differently to previous development projects and how did it work?

%Log: feb9 (git branch setup, later march2 argue if static analysis removes need for manual reviews), feb25 (CI/CD, using pipelines in your projectm (discuss including security scan to workflow)), automating provisioning (docker, vangrant..),  instrumenting, scaling cloud environment + loadbalancing, 


% One aspect of our practices that does not align with the DevOps methodology is the configuration of our virtual machines. 
% While Vagrant was used to automate the provisioning of our initial single-node cloud infrastructure, we did not include any further virtual machine configuration in our automated deployment. 
% As a result, we lacked version control and an overview of the virtual machine's state after its initial setup. 
We created a DevOps anti-pattern where all changes to the virtual machine were made manually via SSH, without version control and traceability when making configuration changes later in the project. 
This led to an erosion of our automated cloud provisioning, deviating more and more from our initial setup, making it increasingly difficult and time-consuming to maintain. Initially we had used Vagrant to provision our single-node cloud infrastructure, but did not maintain it throughout the project.
Ultimately, this led to more time spent on fixing human errors and configuring new virtual machines. 
Additionally, if the infrastructure were to crash, it would take days to reproduce the current state of the service.
%vi skriver ikke noget om TerraForm, vi er bare ærlige til eksamen og siger at vi først satte det op efterfølgende

Besides including further virtual machine configuration to our automated deployment, it would have been beneficial to include a security aspect to our pipeline, namely snyk and wmap to provide reactive identification of vulnerabilities. 
% Specifically, adding both the \textit{snyk} and \textit{wmap} scanning, also used for our security assessment, would provide a reactive identification of vulnerabilities before each deployment.

%Var der noget godt ved devops practices.. herunder;

%Git repo setup... branch protection.. must satisfy some level of linting